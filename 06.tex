
\chapter{展望}
\label{chap:outlook}

今後、簡素化支援システムは様々な技術と掛け合わせをすることで発展していく事が考えられる。例えば、IoTの普及によって、モノがインターネットを通じて操作ができるようになった際には、ボタンやスイッチを簡素化の対象として排除し、AR上の仮想的な操作パネルにて操作を行うようにすることも可能である。また、人工知能によって、仮想的なモノの表示位置を姿勢などから推定したり、シチュエーションに応じて表示するモノを勝手に決定する事などもできる。ARは視界にデジタル情報を加える表示方法でしかないため、技術やコンテンツが今後の重要なポイントになると推察される。

また、簡素化支援システムが普及した場合、モノに対する意識に変化が起こる可能性がある。前述したボタンの簡素化について、これまでは物理的なボタンがあったからこそ、そのモノを操作しているという意識があった。しかし、それが排除された場合、概念上で機能と物的存在に切り分けられ、モノの物的存在への意識が薄れ、いわばクラウドサービスのような、掴みどころのない機能として意識されていく事が予測できる。