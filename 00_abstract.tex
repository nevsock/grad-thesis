% ■ アブストラクトの出力 ■
%	◆書式:
%		begin{jabstract}〜end{jabstract}	:日本語のアブストラクト
%		begin{eabstract}〜end{eabstract}	:英語のアブストラクト
%		※ 不要ならばコマンドごと消せば出力されない。



% 日本語のアブストラクト
\begin{jabstract}

本研究では、生活上に要するモノを、拡張現実技術で仮想的に代替することにより、生活の簡素化と機能充実を両立する、簡素化支援システムを提案する。生活の簡素化は、部屋内のモノがあることによって与えられる不快感や煩わしさの解消に寄与するが、同時にそのモノの持つ機能を生活から排除することとなってしまう。そこで、簡素化の対象として生活から排除したモノをARを用いて代替することで、従来通りの機能を得られるシステムを開発した。これにより、生活の簡素化と機能充実の両立の他、従来以上の機能充実なども期待することが可能となった。

本論文では、簡素化支援システムの提案とその可能性を提示した上で、実際に運用しての評価や、関連システムより導ける用途、今後の展望について考察する。

\end{jabstract}

