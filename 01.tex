\chapter{序論}
\label{chap:introduction}

論文は序論のようなもので始める。タイトルは序論でも序言でもはじめにでもいいけど、『序論』で始めたら『結論』で終わり、『序言』で始めたら『結言』で終わるようにする。『はじめに』なら『おわりに』で終わる。『序論』で始まって『おわりに』でおわるとか、そういうちぐはぐなのはだめ。

ここでは序論として書く。序論では、研究の背景やら目的やらを書くのが普通。今はテンプレートの説明なので、大して書くことは無い。

\newpage

\section{本研究に至る背景}

私たちはモノを使用することで充実した生活を享受している。そのため、自身のニーズに合致するモノを多く持っている程、生活が充実する。しかし、それらが物体であるが故に、生活の邪魔となったり、使用後に片付けなくてはならなかったりと、不快感や煩わしさを引き起こす原因となる事も多々ある。そのため、部屋に配置するモノを必要最小限に抑え、生活を簡素化することは、最近支持を得られるようになってきた。

生活を簡素化するためにモノを排除する際、機能充実を取るか、不快感や煩わしさの解消を取るかのトレードオフとなる。しかし、モノによっては、この二択を選択することが困難な場合も存在する。

そこで、拡張現実技術(AR)を用いることで、簡素化をしながらも、生活に要するモノを仮想的に再現することを考えた。物理的に依存関係を持たないモノであれば、ARを用いて環境を再現することが可能となる。よって、本研究では、部屋内のモノをARで再現し、仮想的に代替することにより、生活の簡素化と機能充実の両方を実現するシステムについて提案し、考察する。

\section{提案目的}

日常的に利用している部屋内の要素を部屋内のモノをARで再現し、仮想的に代替することにより、生活の簡素化と機能充実を両立し、利便化を図ることが本研究の目的である。

\section{本論文の構成}

本論文は、以下の7章で構成される。
2章では、簡素化支援システムの提案と具体例についての考察を行う。3章では、実際に制作した簡素化支援システムについての詳細について説明する。4章では、2章・3章を元に、簡素化支援システムを運用しての考察と問題点について述べる。5章では、関連システムを例示し、簡素化支援システムとの相違点や参考点をまとめる。6章では、これまでの考察を元に、簡素化支援システムの今後の展望について述べる。最後に、7章では、全体を通しての総括を行う。
