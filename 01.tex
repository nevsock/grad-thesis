\chapter{序論}
\label{chap:introduction}

本章では、本研究の背景と本論文の構成について述べる。

\newpage

\section{背景}

私たちの生活する部屋は、テレビや時計など、様々な「モノ」で溢れている。私たちは、モノの持つ機能を使用することで、自身のニーズを満たし、充実した生活を享受している。そのため、モノを多く所持することで、より充実した生活を得ることができる。しかし、部屋内のモノが多い時、置くスペースを必要とすることや、モノを探す手間が掛かること、汚い様相に不快感を受けることなど、様々な問題を生じてしまう。そのため、部屋に配置するモノを必要最小限に抑え、生活を簡素化することは、最近支持を得られるようになってきた。

生活を簡素化するためにモノを排除する際、機能充実を取るか、不快感や煩わしさの解消を取るかのトレードオフとなる。しかし、モノによっては、この二択を選択することが困難な場合も 存在する。

ただし、物理的に依存関係を持たないモノであれば、ARを用いて環境を再現することが可能となる。そこで、拡張現実技術(AR)を用いることで、生活に要するモノを減らして簡素化をし、その分のモノを仮想的に再現することを考えた。本研究では、部屋内のモノをARで再現し、仮想的に代替することにより、生活の簡素化と機能充実の両方を実現するシステムについて提案し、考察する。

\section{提案目的}

日常的に利用している部屋内のモノをARで再現し、仮想的に代替することにより、生活の簡素化と機能充実を両立し、利便化を図ることが本研究の目的である。

\section{本論文の構成}

\ref{chap:suggestion}章では、簡素化支援システムの提案と具体例についての考察を行う。3章では、実際に制作した簡素化支援システムについての詳細について説明する。4章では、2章・3章を元に、簡素化支援システムを運用しての考察と問題点について述べる。5章では、関連システムを例示し、簡素化支援システムとの相違点や参考点をまとめる。6章では、これまでの考察を元に、簡素化支援システムの今後の展望について述べる。最後に、7章では、全体を通しての総括を行う。
