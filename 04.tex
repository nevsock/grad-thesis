
\chapter{システムの評価と考察}
\label{chap:systemEvaluation}

本章では、前章で実装したシステムを実際に運用した上での考察と問題点について述べる。

\newpage

\section{システムを運用しての考察}

システムの運用時、机以外に何も配置していない環境であったため、窮屈さや煩雑さはなかった。その上で、動画や観葉植物、時計などの機能を得られたため、部屋内のモノが少ない状態であっても十分な充実感を得ることができた。また、仰向けの状態で動画を鑑賞したいと考えた際も、マーカーを壁から天井に移動するだけで、楽な体勢にて鑑賞することが出来たため、体験の面では、従来よりも満足度は高いと感じた。今後、マーカーを使用しなくとも空間上に意図した形式でモノを配置することが容易になれば、より満足度の高い鑑賞体験ができるのではないかと推察する。

一方で、デバイスの認識精度やスペック上に問題点があり、AR上のモノを正常に使用できない場面が多々あった。本システム上の問題点に関しては後述するが、AR上のモノという、直観ではどうしても解決できない不具合が発生した際、実世界のモノの不具合よりも強く煩わしさを感じた。AR上での動作が安定するまでの間は、不具合が起こる可能性を加味した運用をすることが重要だと考える。

\section{問題点}

\subsection{マーカーの認識制度}

本システムを運用した際、フレームレートの大幅な低下や位置ずれなどが発生した。これらは主にマーカー形状と運用環境の明るさの二点が原因であると類推される。

本システムに利用したマーカーは、自作のアルゴリズムによって生成された画像であった。ランダムな大きさ・位置・角度の円形ないし四角形の領域を、白黒で交互に塗りつぶすことで生成していた。システム制作に使用したvuforiaのマーカー評価では、最大点を記録していたが、単純図形のみで構成した画像であったため、特徴点の数が少なく、結果的に認識精度の低下に繋がったと推察する。また、画像認識によってコンテンツの表示位置を決定しているため、運用環境での明るさが精度に影響する場合がある。運用した日が曇りであったこと、シーリングライトの光量が少なかったことなどから、認識精度を保てなかった可能性がある事が考えられる。

\subsection{デバイスの物理的問題点}

本システムでは、日常的にARグラスを付け続ける事を前提としているが、デバイスの重さと視野角の狭さは、日常的な生活をする上で大きな負担となる。将来的にARグラスを日常利用する上で、軽量化と視野角の拡大は必要な改善である。
